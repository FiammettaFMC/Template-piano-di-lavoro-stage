%---------------------------------------------------------------------------------------
%	DESCRIPTION OF THE INTERACTION BETWEEN THE STUDENT AND THE INTERNAL TUTOR
%---------------------------------------------------------------------------------------

\section{Interazione tra studente e tutor aziendale}
	A causa del distanziamento sociale imposto per prevenire il diffondersi del virus Covid-19, lo stage verrà svolto per la maggior parte del tempo da remoto, perciò non sarà sempre possibile avere un'interazione diretta con il tutor aziendale.
	\newline
	Tuttavia, attraverso appositi strumenti di comunicazione telematica predisposti dall'azienda, quali Skype e Microsoft Teams, verrà fissato un incontro al termine della giornata al fine di verificare i progressi ottenuti dallo studente. In questo modo sarà possibile monitorare costantemente lo stato di avanzamento del progetto e quindi garantire il raggiungimento degli obiettivi fissati nella pianificazione, risolvendo prontamente eventuali problematiche riscontrate. 
	\newline
	In queste occasioni, in base alla specifica situazione, sarà possibile chiarire ulteriormente gli obiettivi da raggiungere e raffinare o modificare la pianificazione.
	\newline
	In ogni caso, sarà sempre possibile comunicare tramite e-mail, chat testuali o chat vocali in caso di necessità, senza una preventiva pianificazione.
	
	\subsection{Registro delle attività}
		Per tracciare tutte le attività svolte dallo studente durante lo stage sarà necessario riportare un riassunto giornaliero delle stesse, all’interno di un apposito documento condiviso con il tutor aziendale.
		\newline
		Lo strumento scelto per la compilazione e la gestione di tale registro è Documenti Google. Verrà quindi creato un apposito documento, mantenuto e condiviso tramite Google Drive, al cui interno sarà presente una tabella, la quale conterrà un riassunto chiaro e conciso delle attività svolte durante ogni giornata lavorativa del periodo di stage, riportate in ordine cronologico.
		\newline
		Una volta terminato lo stage, tale documento varrà come garanzia dell'impegno orario e del lavoro effettivamente svolto dallo studente, in modo da abilitare la firma del tutor aziendale sul modulo di fine stage.
		\newline
		La registrazione delle attività verrà eseguita rispettando le seguenti regole:
		\begin{itemize}
			\item il documento deve essere aggiornato dallo studente al termine di ogni giornata lavorativa, in orario serale;
			\item ogni aggiornamento deve riportare le seguenti informazioni:
			\begin{itemize}
				\item descrizione delle attività svolte dallo studente;
				\item descrizione di eventuali problemi o difficoltà incontrate nello svolgimento;
			\end{itemize}
			\item il tutor aziendale deve avere la possibilità di visionare le ultime modifiche apportate dallo studente e lasciare dei commenti in merito ad eventuali mancanze od imprecisioni.
		\end{itemize}
		Nonostante il registro sia principalmente oggetto del rapporto tra studente e tutor aziendale, il documento sarà accessibile anche da parte del tutor accademico, il quale potrà interagirci con le stesse modalità sopra citate.

	
