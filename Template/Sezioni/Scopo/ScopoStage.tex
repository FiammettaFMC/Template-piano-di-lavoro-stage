%---------------------------------------------------------------------------------------
%	STAGE DESCRIPTION
%---------------------------------------------------------------------------------------

\subsection{Scopo del progetto}
	\ragioneSocAzienda\ fornisce diversi strumenti per eliminare l'utilizzo di documenti cartacei all’interno delle aziende, in modo da agevolare e tracciare il trasferimento delle informazioni. Oltre a \ragioneSocAzienda, il progetto coinvolge anche la start-up Commerc.io che gestisce lo scambio di documenti siglati da firme elettroniche tramite una propria blockchain.
	\newline
	La blockchain è una insieme di tecnologie che permette il mantenimento di un registro strutturato in una catena di blocchi, contenenti le transazioni, la cui validazione è affidata ad un meccanismo di consenso garantito dall'insieme stesso dei blocchi.
	\newline
	Lo scopo del progetto di stage è definire dei processi che vedano l'integrazione, tramite la blockchain sviluppata da Commerc.io, di identità digitali basate su self-sovereign identity e sulla condivisione e notarizzazione di processi di condivisione e sottoscrizione con firme elettroniche dei documenti.
	\newline
	Con self-sovereign identity si intende un meccanismo che permette agli individui di certificare la propria identità digitale, senza bisogno che un garante esterno lo faccia per loro.
	\newline
	Tutto questo dovrà essere integrato all'interno dei processi di digital transformation di \ragioneSocAzienda.
	\newline
	Lo studente dovrà, inizialmente, effettuare uno studio di fattibilità dettagliato del progetto Commerc.io nel contesto dei documenti fiscali, per poi svilupparne un \textit{proof of concept}. L'attività di sviluppo prevederà, oltre all'analisi iniziale, un periodo di progettazione, il tutto opportunamente argomentato e documentato.
	\newline
	Tuttavia il focus del progetto non sarà il prodotto software funzionante, ma l'attività critica ed analitica preparatoria al suo sviluppo.
	
	TODO: aggiungere ref per progetto di ricerca, in modo da giustificare la pianificazione non troppo dettagliata
