%---------------------------------------------------------------------------------------
%	STAGE DESCRIPTION
%---------------------------------------------------------------------------------------

\subsection{Scopo del progetto}
	\ragioneSocAzienda\ fornisce diversi strumenti per eliminare l'utilizzo di documenti cartacei all’interno delle aziende, in modo da agevolare e tracciare il trasferimento delle informazioni. Oltre a \ragioneSocAzienda, il progetto coinvolge anche la start-up Commerc.io che gestisce lo scambio di documenti siglati da firme elettroniche tramite una propria blockchain.
	\newline
	Lo scopo del progetto di stage è la realizzazione di un applicativo che veda l'integrazione, tramite la blockchain sviluppata da Commerc.io, di identità digitali basate su self-sovereign identity e sulla notarizzazione di processi di condivisione e sottoscrizione, con firme elettroniche, dei documenti fiscali.
	\newline
	Con self-sovereign identity si intende un meccanismo che permette agli individui di certificare la propria identità digitale, senza bisogno che un garante esterno lo faccia per loro.
	\newline
	Tutto questo dovrà essere integrato all'interno dei processi di digital transformation di \ragioneSocAzienda.
	\newline
	Lo studente dovrà, inizialmente, effettuare uno studio ed un'analisi dettagliata della documentazione del progetto Commerc.io nel contesto dei documenti fiscali, in modo da approfondire le tecnologie riguardanti la blockchain. Dovrà poi essere sviluppato un \textit{proof of concept}, che si servirà delle API appositamente implementate da Commerc.io per l'interazione con la loro rete \textit{commercio.network}.
	\newline
	L'attività di sviluppo prevederà, oltre all'analisi iniziale, un periodo di progettazione, che dovrà essere opportunamente argomentata e documentata. Il prototipo finale, risultante dall'attività di codifica, dovrà permettere l'interazione con la blockchain di Commerc.io per il riconoscimento delle identità digitali e la creazione di transazioni per la trasmissione di documenti.