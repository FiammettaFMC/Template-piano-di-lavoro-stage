%---------------------------------------------------------------------------------------
%	OBJECTIVES
%---------------------------------------------------------------------------------------

\section{Obiettivi}
	\subsection{Classificazione}
		Si farà riferimento ai requisiti secondo la seguente classificazione, la quale permette di identificarli univocamente:
		\begin{itemize}
			\item \textbf{RO-X:} requisiti obbligatori, vincolanti in quanto obiettivo primario richiesto dall'azienda proponente;
			\item \textbf{RD-X:} requisiti desiderabili, non vincolanti o strettamente necessari, ma dal riconoscibile valore aggiunto;
			\item \textbf{RF-X:} requisiti facoltativi, rappresentanti valore aggiunto non strettamente competitivo.
		\end{itemize}
		
		Nelle sigle precedentemente indicate, \textit{X} è un numero intero progressivo maggiore di zero con funzione di identificativo del requisito.
	
	\subsection{Definizione}
		Si prevede lo svolgimento dei seguenti obiettivi.

		\subsubsection*{Obbligatori}
		\begin{itemize}
			\obiettiviObbligatori
		\end{itemize}
		
		\subsubsection*{Desiderabili}
		\begin{itemize}
			\obiettiviDesiderabili
		\end{itemize}
		
		\subsubsection*{Facoltativi}
		\begin{itemize}
			\obiettiviFacoltativi
		\end{itemize} 
