\subsection{Contenuti formativi}
Durante questo progetto di stage lo studente avrà occasione di approfondire le sue conoscenze nell'ambito dei processi markoviani e della loro applicazione nel campo delle reti informatiche.
	\newline
	In particolare lo studio si concentrerà su:
	\begin{itemize}
		\item concetti base delle catene di Markov a tempo discreto: definizione, matrice di transizione, classificazione degli stati, distribuzione stazionaria;
		\item algoritmi \textit{Monte Carlo};
		\item acquisizione delle competenze necessarie alla simulazione di una catena di Markov su un grafo
		\item sviluppo della capacità di valutazione delle prestazioni registrate dall'applicazione dell'algoritmo come strumento di ricerca o navigazione.
	\end{itemize}
