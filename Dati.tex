%----------------------------------------------------------------------------------------
%   USEFUL COMMANDS
%----------------------------------------------------------------------------------------

\newcommand{\dipartimento}{Dipartimento di Matematica "Tullio Levi-Civita''}

%----------------------------------------------------------------------------------------
% 	USER DATA
%----------------------------------------------------------------------------------------

% Data di approvazione del piano da parte del tutor interno; nel formato GG Mese AAAA
% compilare inserendo al posto di GG 2 cifre per il giorno, e al posto di 
% AAAA 4 cifre per l'anno
\newcommand{\dataApprovazione}{Data}

% Dati dello Studente
\newcommand{\nomeStudente}{Fiammetta}
\newcommand{\cognomeStudente}{Cannavò}
\newcommand{\matricolaStudente}{1167703}
\newcommand{\emailStudente}{fiammetta.cannavo@studenti.unipd.it}


% Dati del Docente Proponente (Docente)
\newcommand{\titoloDocenteProponente}{Dott.ssa}
\newcommand{\nomeDocenteProponente}{Francesca}
\newcommand{\cognomeDocenteProponente}{Collet}
\newcommand{\emailDocenteProponente}{fcollet@math.unipd.it}

% Dati del Tutor Interno (Docente)
\newcommand{\titoloTutorInterno}{Professor}
\newcommand{\nomeTutorInterno}{Alessandro}
\newcommand{\cognomeTutorInterno}{Sperduti}
\newcommand{\emailTutorInterno}{sperduti@math.unipd.it}

\newcommand{\prospettoSettimanale}{

    \subsubsection{Settimane I, II: catene di Markov}
    \textbf{Obiettivo quindicinale:} acquisizione delle nozioni fondamentali sulle catene di Markov a tempo discreto, omogenee e a stati finiti.
    \\
    \textbf{Ore di lavoro:} 80.
    
    \paragraph{Settimana I - 40 ore}   
    \begin{itemize}
    	\item Studio di definizioni e risultati principali relativi ad una catena di Markov:
    	\begin{itemize}
    		\item matrice di transizione, proprietà di Markov e distribuzione;
    		\item transienza e ricorrenza;
    		\item irriducibilità e aperiodicità;
    		\item passeggiate aleatorie su grafi.
    	\end{itemize}
    \end{itemize}

    \paragraph{Settimana II - 40 ore}
  
    \begin{itemize}
        \item Studio del comportamento asintotico di una catena di Markov:
       \begin{itemize}
       		\item distribuzione stazionaria;
       		\item convergenza all’equilibrio;
       		\item algoritmi \textit{Monte Carlo} per la simulazione di catene di Markov su grafi.
       \end{itemize} 
		\item Redazione di una breve relazione che riassuma i concetti e risultati chiave
visti durante le settimane I, II.
    \end{itemize}

    \subsubsection{Settimane III, IV, V, VI: esplorazione di network}
    \textbf{Obiettivo mensile}: implementazione di un algoritmo di ricerca/navigazione basato su passeggiate aleatorie.
    \\
    \textbf{Ore di lavoro:} 160.
    
    \paragraph{Settimana III - 40 ore} 

    \begin{itemize}
    	\item Studio dei contenuti modellistici degli articoli:
    	\begin{itemize}
    		\item Riascos and Mateos. \textit{Long-range navigation on complex networks using Lévy
random walks.} Phys. Rev. E \textbf{86}, 056110 (2012);
			\item Weng et al. \textit{Navigation by anomalous random walks on complex networks.}
Sci. Rep. \textbf{6}, 37547 (2016).
    	\end{itemize}
    \end{itemize}

    \paragraph{Settimana IV - 40 ore} 
    
    \begin{itemize}
        \item Implementazione di un algoritmo per la simulazione di una passeggiata aleatoria (anomala) su un grafo.
    \end{itemize}

    \paragraph{Settimana V - 40 ore} 
    
    \begin{itemize}
        \item \textsc{[continuazione]} Implementazione di un algoritmo per la simulazione di una passeggiata aleatoria (anomala) su un grafo.
		\item Redazione di una breve relazione che riassuma i contenuti degli articoli studiati durante la settimana III e riporti e spieghi gli algoritmi sviluppati.
    \end{itemize}

    \paragraph{Settimana VI - 40 ore} 
    
    \begin{itemize}
        \item Esecuzione di test per il confronto dell’efficienza di navigazione attraverso passeggiate aleatorie semplici e anomale.
		\item Applicazione all’algoritmo di ricerca \textit{PageRank}.
		\item Redazione di una breve relazione che contenga un riassunto esaustivo dei test effettuati e dei relativi risultati.
    \end{itemize}

    \subsubsection{Settimane VII, VIII: relazione finale} 
    \textbf{Obiettivo quindicinale:} considerazioni finali e stesura della documentazione conclusiva.
    \\
    \textbf{Ore di lavoro:} 80.
    
    \paragraph{Settimana VII - 40 ore}  
    
    \begin{itemize}
        \item Discussione dei risultati ottenuti e redazione di una breve relazione di considerazioni finali su tali risultati.
    \end{itemize}

    \paragraph{Settimana VIII - 40 ore}  
    
    \begin{itemize}
    	 \item Raccolta della documentazione prodotta durante le settimane I-VII e redazione di una versione preliminare della relazione finale.
    \end{itemize}
    
    \subsection{Estremi temporali}
    \textbf{Data di inizio del progetto:} 13/07/2020. \\
    \textbf{Data di fine del progetto:} 04/09/2020.
}

% Indicare il totale complessivo (deve essere compreso tra le 300 e le 320 ore)
\newcommand{\totaleOre}{320}

\newcommand{\obiettiviObbligatori}{
	 \item Approfondimento delle catene di Markov a tempo discreto, omogenee e a stati finiti.
	 \item Implementazione di una classe di algoritmi di ricerca/navigazione su classi di \textit{complex network} notevoli.
}

\newcommand{\obiettiviDesiderabili}{
	 \item Applicazione dell'algoritmo sviluppato all’algoritmo di ricerca \textit{PageRank}.
}

\newcommand{\obiettiviFacoltativi}{
	 \item Implementazione di una classe di algoritmi di ricerca/navigazione su classi di \textit{complex network} reali come \textit{web-Standford} o \textit{Ego-Facebook}.
}
