%----------------------------------------------------------------------------------------
%   USEFUL COMMANDS
%----------------------------------------------------------------------------------------

\newcommand{\dipartimento}{Dipartimento di Matematica ``Tullio Levi-Civita''}

%----------------------------------------------------------------------------------------
% 	USER DATA
%----------------------------------------------------------------------------------------

% Data di approvazione del piano da parte del tutor interno; nel formato GG Mese AAAA
% compilare inserendo al posto di GG 2 cifre per il giorno, e al posto di 
% AAAA 4 cifre per l'anno
\newcommand{\dataApprovazione}{Data}

% Dati dello Studente
\newcommand{\nomeStudente}{Fiammetta}
\newcommand{\cognomeStudente}{Cannavò}
\newcommand{\matricolaStudente}{1167703}
\newcommand{\emailStudente}{fiammetta.cannavo@studenti.unipd.it}


% Dati del Docente Proponente (Docente)
\newcommand{\titoloDocenteProponente}{Professoressa}
\newcommand{\nomeDocenteProponente}{Francesca}
\newcommand{\cognomeDocenteProponente}{Collet}
\newcommand{\emailDocenteProponente}{fcollet@math.unipd.it}

% Dati del Tutor Interno (Docente)
\newcommand{\titoloTutorInterno}{Professor}
\newcommand{\nomeTutorInterno}{}
\newcommand{\cognomeTutorInterno}{}
\newcommand{\emailTutorInterno}{}

\newcommand{\prospettoSettimanale}{

    \subsubsection{Prima settimana}
    \textbf{Obiettivo quindicinale:} acquisizione delle nozioni fondamentali sulle catene di Markov a tempo discreto, omogenee e a stati finiti.

    \begin{itemize}
    	\item 
    \end{itemize}

    \subsubsection{Seconda settimana} 
    \begin{itemize}
        \item 
    \end{itemize}

    \subsubsection{Terza settimana}
    \begin{itemize}
    	\item 
    \end{itemize}

    \subsubsection{Quarta settimana} 
    \begin{itemize}
        \item 
    \end{itemize}

    \subsubsection{Quinta settimana} 
    \begin{itemize}
        \item 
    \end{itemize}

    \subsubsection{Sesta settimana} 
    \begin{itemize}
        \item
    \end{itemize}

    \subsubsection{Settima settimana} 
    \begin{itemize}
        \item 
    \end{itemize}

    \subsubsection{Ottava settimana} 
    \begin{itemize}
        \item 
    \end{itemize}
}

% Indicare il totale complessivo (deve essere compreso tra le 300 e le 320 ore)
\newcommand{\totaleOre}{320}

\newcommand{\obiettiviObbligatori}{
	 \item approfondimento delle catene di Markov a tempo discreto, omogenee e a stati finiti;
	 \item implementazione di una classe di algoritmi di ricerca/navigazione su classi di \textit{complex network} notevoli;
}

\newcommand{\obiettiviDesiderabili}{
	 \item applicazione all’algoritmo di ricerca \textit{PageRank}.
}

\newcommand{\obiettiviFacoltativi}{
	 \item implementazione di una classe di algoritmi di ricerca/navigazione su classi di \textit{complex network} reali come \textit{web-Standford} o \textit{Ego-Facebook}.
}
