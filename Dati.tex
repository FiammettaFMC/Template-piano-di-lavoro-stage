%----------------------------------------------------------------------------------------
%   USEFUL COMMANDS
%----------------------------------------------------------------------------------------

\newcommand{\dipartimento}{Dipartimento di Matematica ``Tullio Levi-Civita''}

%----------------------------------------------------------------------------------------
% 	USER DATA
%----------------------------------------------------------------------------------------

% Data di approvazione del piano da parte del tutor interno; nel formato GG Mese AAAA
% compilare inserendo al posto di GG 2 cifre per il giorno, e al posto di 
% AAAA 4 cifre per l'anno
\newcommand{\dataApprovazione}{Data}

% Dati dello Studente
\newcommand{\nomeStudente}{Lorenzo}
\newcommand{\cognomeStudente}{Dei Negri}
\newcommand{\matricolaStudente}{1161729}
\newcommand{\emailStudente}{lorenzo.deinegri@studenti.unipd.it}
\newcommand{\telStudente}{+ 39 346 371 2289}

% Dati del Tutor Aziendale
\newcommand{\nomeTutorAziendale}{Andrea}
\newcommand{\cognomeTutorAziendale}{Bisello}
\newcommand{\emailTutorAziendale}{andrea.bi@2csolution.it}
\newcommand{\telTutorAziendale}{+ 39 049 942 6171}
\newcommand{\ruoloTutorAziendale}{}

% Dati dell'Azienda
\newcommand{\ragioneSocAzienda}{2C Solution S.r.l.}
\newcommand{\indirizzoAzienda}{Via Martin Piva Artigiano 12, Limena (PD)}
\newcommand{\sitoAzienda}{https://www.2csolution.it/}
\newcommand{\emailAzienda}{info@2csolution.it}
\newcommand{\partitaIVAAzienda}{P.IVA IT04030410288}

% Dati del Tutor Interno (Docente)
\newcommand{\titoloTutorInterno}{Prof.}
\newcommand{\nomeTutorInterno}{Massimo}
\newcommand{\cognomeTutorInterno}{Marchiori}

\newcommand{\prospettoSettimanale}{
     % Personalizzare indicando in lista, i vari task settimana per settimana
     % sostituire a XX il totale ore della settimana
    \subsubsection{Prima Settimana (XX ore)}
    \begin{itemize}
        \item Orientamento in azienda:
        \begin{itemize}
        	\item strumenti in uso;
          	\item contesto della fatturazione elettronica;
        \end{itemize}
        \item studio e ricerca:
        \begin{itemize}
        	\item concetti di blockchain nell'ambito della trasmissione di documenti di natura fiscale (fatture).
        \end{itemize}
    \end{itemize}

    \subsubsection{Seconda Settimana - Sottotitolo (XX ore)} 
    \begin{itemize}
        \item Orientamento in azienda:
     	\begin{itemize}
           	\item strumenti in uso;
           	\item contesto della fatturazione elettronica;
     	\end{itemize}
        \item studio e ricerca:
      	\begin{itemize}
           	\item concetti di blockchain nell'ambito della trasmissione di documenti di natura fiscale (fatture).
      	\end{itemize}
    \end{itemize}

    \subsubsection{Terza Settimana - Sottotitolo (XX ore)} 
    \begin{itemize}
        \item Operatività sulla rete blockchain commercio.network tramite l'sdk:
        \begin{itemize}
           	\item connessione alla rete;
           	\item creare transazioni;
           	\item firmare transazioni;
           	\item diventare parte della rete;
           	\item connessione con altre reti.
        \end{itemize}
    \end{itemize}

    \subsubsection{Quarta Settimana - Sottotitolo (XX ore)} 
    \begin{itemize}
        \item Operatività sulla rete blockchain commercio.network tramite l'sdk:
        \begin{itemize}
           	\item connessione alla rete;
           	\item creare transazioni;
           	\item firmare transazioni;
           	\item diventare parte della rete;
           	\item connessione con altre reti.
        \end{itemize}
    \end{itemize}

    \subsubsection{Quinta Settimana - Sottotitolo (XX ore)} 
    \begin{itemize}
        \item Operatività sulla rete blockchain commercio.network tramite l'sdk:
        \begin{itemize}
           	\item connessione alla rete;
           	\item creare transazioni;
           	\item firmare transazioni;
           	\item diventare parte della rete;
           	\item connessione con altre reti.
        \end{itemize}
    \end{itemize}

    \subsubsection{Sesta Settimana - Sottotitolo (XX ore)} 
    \begin{itemize}
        \item Operatività sulla rete blockchain commercio.network tramite l'sdk:
        \begin{itemize}
           	\item connessione alla rete;
           	\item creare transazioni;
           	\item firmare transazioni;
           	\item diventare parte della rete;
           	\item connessione con altre reti.
        \end{itemize}
    \end{itemize}

    \subsubsection{Settima Settimana - Sottotitolo (XX ore)} 
    \begin{itemize}
        \item Documentazione aziendale;
        \item documentazione stage;
        \item presentazione in azienda.
    \end{itemize}

    \subsubsection{Ottava Settimana - Conclusione (XX ore)} 
    \begin{itemize}
        \item Documentazione aziendale;
        \item documentazione stage;
        \item presentazione in azienda.
    \end{itemize}
}

% Indicare il totale complessivo (deve essere compreso tra le 300 e le 320 ore)
\newcommand{\totaleOre}{300}

\newcommand{\obiettiviObbligatori}{
	 \item \textbf{RO-1:} acquisizione delle competenze sulle tematiche relative alla blockchain e alle tecnologie associate;
	 \item \textbf{RO-2:} implementazione di un \textit{proof of concept} che permetta l'interazione con la blockchain di Commerc.io per il riconoscimento delle identità digitali e la trasmissione di documenti, tramite un'interfaccia grafica basilare;
	 \item \textbf{RO-3:} stesura della documentazione relativa ai risultati ottenuti nel corso delle varie fasi del progetto e alla gestione delle difficoltà incontrate;
	 \item \textbf{RO-4:} stesura della realzione riguardante il lavoro avor svolto durante tutto il periodo di stage.
}

\newcommand{\obiettiviDesiderabili}{
	 \item \textbf{RD-1:} implementazione di un'interfaccia web completa per il \textit{proof of concept}.
}

\newcommand{\obiettiviFacoltativi}{
	 \item \textbf{RF-1:} sviluppo di un'applicazione mobile in Flutter che offra le funzionalità richieste per il \textit{proof of concept}.
}
