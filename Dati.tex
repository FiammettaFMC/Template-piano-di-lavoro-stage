%----------------------------------------------------------------------------------------
%   USEFUL COMMANDS
%----------------------------------------------------------------------------------------

\newcommand{\dipartimento}{Dipartimento di Matematica ``Tullio Levi-Civita''}

%----------------------------------------------------------------------------------------
% 	USER DATA
%----------------------------------------------------------------------------------------

% Data di approvazione del piano da parte del tutor interno; nel formato GG Mese AAAA
% compilare inserendo al posto di GG 2 cifre per il giorno, e al posto di 
% AAAA 4 cifre per l'anno
\newcommand{\dataApprovazione}{Data}

% Dati dello Studente
\newcommand{\nomeStudente}{Lorenzo}
\newcommand{\cognomeStudente}{Dei Negri}
\newcommand{\matricolaStudente}{1161729}
\newcommand{\emailStudente}{lorenzo.deinegri@studenti.unipd.it}
\newcommand{\telStudente}{+ 39 346 371 2289}

% Dati del Tutor Aziendale
\newcommand{\nomeTutorAziendale}{Andrea}
\newcommand{\cognomeTutorAziendale}{Bisello}
\newcommand{\emailTutorAziendale}{andrea.bi@2csolution.it}
\newcommand{\telTutorAziendale}{+ 39 049 942 6171}
\newcommand{\ruoloTutorAziendale}{}

% Dati dell'Azienda
\newcommand{\ragioneSocAzienda}{2C Solution S.r.l.}
\newcommand{\indirizzoAzienda}{Via Martin Piva Artigiano 12, Limena (PD)}
\newcommand{\sitoAzienda}{https://www.2csolution.it/}
\newcommand{\emailAzienda}{info@2csolution.it}
\newcommand{\partitaIVAAzienda}{P.IVA IT04030410288}

% Dati del Tutor Interno (Docente)
\newcommand{\titoloTutorInterno}{Prof.}
\newcommand{\nomeTutorInterno}{Massimo}
\newcommand{\cognomeTutorInterno}{Marchiori}

\newcommand{\prospettoSettimanale}{
     % Personalizzare indicando in lista, i vari task settimana per settimana
     % sostituire a XX il totale ore della settimana
    \subsubsection{Prima settimana}
    \begin{itemize}
    	\item studio dei concetti fondamentali legati alla blockchain;
    	\item formazione riguardo il contesto aziendale della fatturazione elettronica;
    	\item studio della documentazione della blockchain sviluppata da Commerc.io.
    \end{itemize}

    \subsubsection{Seconda settimana} 
    \begin{itemize}
        \item studio della funzione di un account riferito alla blockchain di Commerc.io, con particolare attenzione al concetto di wallet;
        \item studio della creazione e del mantenimento di un'identità digitale all'interno della blockchain e delle sue possibili interazioni con essa;
        \item studio delle API sviluppate da Commerc.io per la gestione dei documenti tramite la blockchain.
    \end{itemize}

    \subsubsection{Terza settimana}
    \begin{itemize}
    	\item implementazione delle seguenti funzionalità, relative ad un account nella blockchain di Commerc.io:
    	\begin{itemize}
    		\item generazione ed importazione di un HD Wallet;
    		\item richiesta, ricezione ed invio di Token tra diversi account, all'interno della blockchain;
    		\item controllo del bilancio di uno specifico account;
    		\item acquisizione e perdita dello stato di Validator per i Token utilizzati nella blockchain;
    	\end{itemize}
    	\item verifica delle funzionalità implementate.
    \end{itemize}

    \subsubsection{Quarta settimana} 
    \begin{itemize}
        \item implementazione delle seguenti funzionalità, relative alla gestione dell'identità digitale nella blockchain di Commerc.io:
        \begin{itemize}
        	\item utilizzo e gestione di DID (Decentralized Identifier);
        	\item funzioni legate ad un DID: creazione di un DDO (DID Document), richiesta di deposito e richiesta di power up;
        	\item gestione degli inviti ad una connessione;
        	\item gestione delle credenziali verificabili;
        \end{itemize}
    	\item verifica delle funzionalità implementate.
    \end{itemize}

    \subsubsection{Quinta settimana} 
    \begin{itemize}
        \item implementazione delle seguenti funzionalità, relative alla gestione e allo scambio di documenti nella blockchain di Commerc.io:
        \begin{itemize}
        	\item condivisione di un documento;
        	\item invio di una ricevuta;
        	\item gestione delle liste di documenti e ricevute;
        \end{itemize}
   		\item verifica delle funzionalità implementate.
    \end{itemize}

    \subsubsection{Sesta settimana} 
    \begin{itemize}
        \item implementazione dell'interfaccia grafica per la fruizione delle funzionalità implementate da parte degli utenti;
        \item validazione del \textit{proof of concept} e verifica dell'integrazione con l'interfaccia.
    \end{itemize}

    \subsubsection{Settima settimana} 
    \begin{itemize}
        \item stesura della documentazione relativa a quanto sviluppato;
        \item inizio stesura della relazione sull'attività di stage.
    \end{itemize}

    \subsubsection{Ottava settimana} 
    \begin{itemize}
        \item fine stesura della relazione sull'attività di stage;
        \item preparazione della presentazione sull'attività di stage da fare in azienda.
    \end{itemize}
}

% Indicare il totale complessivo (deve essere compreso tra le 300 e le 320 ore)
\newcommand{\totaleOre}{300}

\newcommand{\obiettiviObbligatori}{
	 \item \textbf{RO-1:} acquisizione delle competenze sulle tematiche relative alla blockchain e alle tecnologie associate;
	 \item \textbf{RO-2:} implementazione di un \textit{proof of concept} che permetta l'interazione con la blockchain di Commerc.io per il riconoscimento delle identità digitali e la trasmissione di documenti, tramite un'interfaccia grafica basilare;
	 \item \textbf{RO-3:} stesura della documentazione relativa ai risultati ottenuti nel corso delle varie fasi del progetto e alla gestione delle difficoltà incontrate;
	 \item \textbf{RO-4:} stesura della realzione riguardante il lavoro avor svolto durante tutto il periodo di stage.
}

\newcommand{\obiettiviDesiderabili}{
	 \item \textbf{RD-1:} implementazione di un'interfaccia web completa per il \textit{proof of concept}.
}

\newcommand{\obiettiviFacoltativi}{
	 \item \textbf{RF-1:} sviluppo di un'applicazione mobile in Flutter che offra le funzionalità richieste per il \textit{proof of concept}.
}
