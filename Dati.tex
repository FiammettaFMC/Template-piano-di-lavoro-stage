%----------------------------------------------------------------------------------------
%   USEFUL COMMANDS
%----------------------------------------------------------------------------------------

\newcommand{\dipartimento}{Dipartimento di Matematica ``Tullio Levi-Civita''}

%----------------------------------------------------------------------------------------
% 	USER DATA
%----------------------------------------------------------------------------------------

% Data di approvazione del piano da parte del tutor interno; nel formato GG Mese AAAA
% compilare inserendo al posto di GG 2 cifre per il giorno, e al posto di 
% AAAA 4 cifre per l'anno
\newcommand{\dataApprovazione}{Data}

% Dati dello Studente
\newcommand{\nomeStudente}{Lorenzo}
\newcommand{\cognomeStudente}{Dei Negri}
\newcommand{\matricolaStudente}{1161729}
\newcommand{\emailStudente}{lorenzo.deinegri@studenti.unipd.it}
\newcommand{\telStudente}{+ 39 346 371 2289}

% Dati del Tutor Aziendale
\newcommand{\nomeTutorAziendale}{Andrea}
\newcommand{\cognomeTutorAziendale}{Bisello}
\newcommand{\emailTutorAziendale}{andrea.bi@2csolution.it}
\newcommand{\telTutorAziendale}{+ 39 049 942 6171}
\newcommand{\ruoloTutorAziendale}{}

% Dati dell'Azienda
\newcommand{\ragioneSocAzienda}{2C Solution S.r.l.}
\newcommand{\indirizzoAzienda}{Via Martin Piva Artigiano 12, Limena (PD)}
\newcommand{\sitoAzienda}{https://www.2csolution.it/}
\newcommand{\emailAzienda}{info@2csolution.it}
\newcommand{\partitaIVAAzienda}{P.IVA IT04030410288}

% Dati del Tutor Interno (Docente)
\newcommand{\titoloTutorInterno}{Prof.}
\newcommand{\nomeTutorInterno}{Massimo}
\newcommand{\cognomeTutorInterno}{Marchiori}

\newcommand{\prospettoSettimanale}{
     % Personalizzare indicando in lista, i vari task settimana per settimana
     % sostituire a XX il totale ore della settimana
    \begin{itemize}
        \item \textbf{Prima Settimana (XX ore)}
        \begin{itemize}
            \item Orientamento in azienda:
           	\begin{itemize}
           		\item strumenti in uso;
           		\item contesto della fatturazione elettronica;
           	\end{itemize}
            \item studio e ricerca:
           	\begin{itemize}
           		\item concetti di blockchain nell'ambito della trasmissione di documenti di natura fiscale (fatture).
           	\end{itemize}
        \end{itemize}
        \item \textbf{Seconda Settimana - Sottotitolo (XX ore)} 
        \begin{itemize}
            \item Orientamento in azienda:
	        \begin{itemize}
	           	\item strumenti in uso;
	           	\item contesto della fatturazione elettronica;
	        \end{itemize}
            \item studio e ricerca:
	         \begin{itemize}
	           	\item concetti di blockchain nell'ambito della trasmissione di documenti di natura fiscale (fatture).
	         \end{itemize}
        \end{itemize}
        \item \textbf{Terza Settimana - Sottotitolo (XX ore)} 
        \begin{itemize}
            \item Operatività sulla rete blockchain commercio.network tramite l'sdk:
            \begin{itemize}
            	\item connessione alla rete;
            	\item creare transazioni;
            	\item firmare transazioni;
            	\item diventare parte della rete;
            	\item connessione con altre reti.
            \end{itemize}
        \end{itemize}
        \item \textbf{Quarta Settimana - Sottotitolo (XX ore)} 
        \begin{itemize}
            \item Operatività sulla rete blockchain commercio.network tramite l'sdk:
            \begin{itemize}
            	\item connessione alla rete;
            	\item creare transazioni;
            	\item firmare transazioni;
            	\item diventare parte della rete;
            	\item connessione con altre reti.
            \end{itemize}
        \end{itemize}
        \item \textbf{Quinta Settimana - Sottotitolo (XX ore)} 
        \begin{itemize}
            \item Operatività sulla rete blockchain commercio.network tramite l'sdk:
            \begin{itemize}
            	\item connessione alla rete;
            	\item creare transazioni;
            	\item firmare transazioni;
            	\item diventare parte della rete;
            	\item connessione con altre reti.
            \end{itemize}
        \end{itemize}
        \item \textbf{Sesta Settimana - Sottotitolo (XX ore)} 
        \begin{itemize}
            \item Operatività sulla rete blockchain commercio.network tramite l'sdk:
            \begin{itemize}
            	\item connessione alla rete;
            	\item creare transazioni;
            	\item firmare transazioni;
            	\item diventare parte della rete;
            	\item connessione con altre reti.
            \end{itemize}
        \end{itemize}
        \item \textbf{Settima Settimana - Sottotitolo (XX ore)} 
        \begin{itemize}
            \item Documentazione aziendale;
            \item documentazione stage;
            \item presentazione in azienda.
        \end{itemize}
        \item \textbf{Ottava Settimana - Conclusione (XX ore)} 
        \begin{itemize}
            \item Documentazione aziendale;
            \item documentazione stage;
            \item presentazione in azienda.
        \end{itemize}
    \end{itemize}
}

% Indicare il totale complessivo (deve essere compreso tra le 300 e le 320 ore)
\newcommand{\totaleOre}{}

\newcommand{\obiettiviObbligatori}{
	 \item \textbf{RO-1:} primo obiettivo;
	 \item \textbf{RO-2:} secondo obiettivo;
	 \item \textbf{RO-3:} terzo obiettivo.
}

\newcommand{\obiettiviDesiderabili}{
	 \item \textbf{RD-1:} primo obiettivo;
	 \item \textbf{RD-2:} secondo obiettivo.
}

\newcommand{\obiettiviFacoltativi}{
	 \item \textbf{RF-1:} primo obiettivo;
	 \item \textbf{RF-2:} secondo obiettivo;
	 \item \textbf{RF-3:} terzo obiettivo.
}
